\section{Outline}

\subsection{Introduction}

Rewrite once the rest is done. Probably we can give explicitly how to construct vector fields on odd spheres, the parallelisations for $n=1,3,7$ and describe the connection with $\C, \HH, \OO$.

\subsection{Vector Fields}

\begin{itemize}
\item Euclidean Vector fields as maps \(X : \R \to \R \times \R\), of the form \(x \mapsto (x, \tilde{X}(x))\). Globally trivial vector bundle. Maybe some pictures and examples - e.g. various force fields, rotation fields etc.
\item Vector fields on the sphere are just vector fields \(X\) on \(\R^{n+1}\) such that \(X(x) \perp x\).
\item Construction of non-vanishing fields on odd dimensional spheres
\item Milnor's proof of non-existence for even dimensional spheres
\end{itemize}

\subsection{Division Algebras}

\begin{itemize}
\item What they are
\item Some classification like Hurewitz theorem and Ingelstam's theorem (c.f. proof by John Frolich)
\item Hopf's theorem about even dimension?
\item Milnors theorem classifying?
\item Constructing frames from division algebras
\end{itemize}

\subsection{Manifolds and Vector Bundles}

The usual stuff here.


\subsection{Poincar\'e Hopf and Topological Complexity}

Maybe we can do some counting theorems here - there's a notion of Gauss map and degree of vector bundles. Counting the number of non-vanishing vector fields gives a topological invariant. Also we should do contractibility implies all vector bundles are trivial. Then this leads to homotopy invariance. Now we're really doing differential topology!

\subsection{Hopf Maps}

This seems to fit nicely with everything else, so why not? There are lots of open questions regarding the homotopy groups of spheres.

\subsection{The Missing Cases}

Some discussion here about the odd dimensions not equal to $1,3,7$. Would be amazing if we could give a treatment but I think it might be a bit much to do properly, but who knows... It may be possible to do K-theory on the sphere in an easier way. Adams, Milnor, etc. were also thinking about bigger problems than just the sphere so maybe it can be done with less machinery? Maybe this is for volume 2 :)

\subsection{Further Problems}

There are some very interesting related problems. For example, all compact three-manifolds turn out to be parallelisable. This is proven using characteristic classes. Again, the general theory might be too much but can it be done more easily on the sphere? Maybe this is also for volume 2 :)
